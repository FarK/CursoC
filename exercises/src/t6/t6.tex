\documentclass{tareas}

\Tema{6}

\begin{document}
\maketitle
%TODO: doble página
\newpage
% \phantomsection
% \addcontentsline{toc}{section}{Índice}
\tableofcontents
\phantomsection
\newpage

\section{Cabeceras y funciones}

\begin{problem}[Arrays multidimensionales 1]
\begin{statement}
	\subp Crea una estructura, \lstinline[language=C]|struct array|, que
	guarde el número de columnas, el número de filas y el puntero a un array
	de dos dimensiones.
	\subp Crea un método \lstinline[language=C]|array_alloc(int rows, int cols)|,
	que reserve la memoria para el tamaño indicado y devuelva el
	puntero al array. Utiliza la \textbf{Forma 1} exlicada en clase.
	\subp Crea un método \lstinline[language=C]|array_free| que libere el
	array
	\subp Inicializa e imprime el array
	\subp Comprueba que no hay leaks de memoria con valgrind
\end{statement}
\end{problem}

\begin{problem}[Arrays multidimensionales 2]
\begin{statement}
	\subp Crea una estructura, \lstinline[language=C]|struct array|, que
	guarde el número de columnas, el número de filas y el puntero a un array
	de dos dimensiones.
	\subp Crea un método \lstinline[language=C]|array_alloc(int rows, int cols)|,
	que reserve la memoria para el tamaño indicado y devuelva el
	puntero al array. Utiliza la \textbf{Forma 2} exlicada en clase.
	\subp Crea un método \lstinline[language=C]|array_free| que libere el
	array
	\subp Crea los métodos \lstinline[language=C]|array_set(int i, int j)| y
	\lstinline[language=C]|array_get(int i, int j)| que permitan acceder y
	modificar un elemento en articular de la matriz.
	\subp Inicializa e imprime el array
	\subp Comprueba que no hay leaks de memoria con valgrind
\end{statement}
\end{problem}

\end{document}
