\documentclass{tareas}

\Tema{4}

\begin{document}
\maketitle
%TODO: doble página
\newpage
% \phantomsection
% \addcontentsline{toc}{section}{Índice}
\tableofcontents
\phantomsection
\newpage

\section{Estructuras y enumerados}
\begin{problem}[Person]
\begin{statement}
	Crea una estructura ``\texttt{person}'' como la del ejemplo, un enum
	``\texttt{person\_attr}'' y las siguientes tareas:

	\subp Imprime todos los parámetros de la estructura
	\subp En función del enumerado, imprime el atributo seleccionado
\end{statement}
\end{problem}

\begin{problem}[Pahole]
\begin{statement}
	\textbf{Pahole} es una herramienta para inspeccionar las estructuras que
	usa un programa en C, su tamaño, el de sus elementos, los huecos entre
	ellos, etc.

	\subp Instala la herramienta con \lstinline[language=sh]|sudo apt-get install pahole|
	\subp Compila tu programa con \textbf{símbolos de depuración}
	(\lstinline[language=sh]|gcc -g person.c -o person|)
	\subp Ejecuta pahole sobre el programa: \lstinline[language=sh]|pahole person|
\end{statement}
\end{problem}


\begin{problem}[Car]
\begin{statement}
	Crea una estructura ``\texttt{car}'' con los campos que creas
	convenientes para definir a un coche. Uno de los campos ha de ser
	``\texttt{owner}'' del tipo ``\texttt{struct person}''.

	\subp Imprime los atributos del coche y los de su dueño
\end{statement}
\end{problem}

\clearpage
\section{Macros, uniones y campos de bits}
\begin{problem}[¿String o int?]
\begin{statement}
	Crea una estructura que permita interpretar un dato de 64 bits como un
	entero sin signo o una cadena de caracteres.
\end{statement}
\end{problem}

\begin{problem}[Mayusculas y errores]
\begin{statement}
	Crea una macro que convierta un caracter
	\href{https://es.wikipedia.org/wiki/ASCII#Caracteres_imprimibles_ASCII}{ASCII}
	a mayúsculas. Si el caracter recibido no es una letra, se debe mostrar
	un error indicando la el nombre del fichero y la fila actuales.
	\\[1em]
	\textbf{Pista}: Las macros de varias lineas deben encerrarse en un
	bloque \lstinline[language=C]|do { ... } while(0);|, para que al
	utilizarlas en un bucle sin llaves se ejecute correctamente. Las lineas
	se finalizan con una barra invertida: \texttt{\textbackslash}.
	\begin{lstlisting}[basicstyle=\normalfont, frame=none, numbers=none, language=C]
		#define PRINT(a, b) do {         \
		        printf("a = %d\n", (a)); \
		        printf("b = %d\n", (b)); \
		} while(0)
	\end{lstlisting}
\end{statement}
\end{problem}

\begin{problem}[Haciendo chorradas]
\begin{statement}
	Usando la estructura y la macro anteriores, realiza un pequeño programa
	que ejecute por orden:

	\subp Inicializa la estructura con caracteres
	\subp Imprime la estructura como una cadena y un entero
	\subp Cambia a mayúsculas tres caracteres cualesquiera de la cadena
	\subp Imprime de nuevo la cadena y el número
	\subp Asigna un número cualquiera a la estructura
	\subp Intenta cambiar un caracter cualquiera para comprobar que se
	imprime el error
\end{statement}
\end{problem}

\end{document}
