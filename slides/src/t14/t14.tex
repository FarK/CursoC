\documentclass{mybeamer}
\usepackage{marvosym}

\institute{
	{\textsl{\large Tema 14}}
	\\[1em]
	\textbf{\Large GTK}
}

\begin{document}
\begin{frame}
\titlepage
\end{frame}

\begin{frame}
\frametitle{Índice}
% \begin{multicols}{2}
	\tableofcontents
% \end{multicols}
\end{frame}

\begin{framesec}{Sobre GTK}
	\begin{itemize}
		\item \textbf{GTK} (Gimp ToolKit) es una biblioteca libre y
			multiplataforma para el desarrollo de interfaces
			gráficas de usuario
		\item Está escrita en C, aunque tiene \textit{bindings} a una
			infinidad de lenguajes distintos (C++, C\#, Python, Lua,
			Haskel, \ldots )
		\item Se trata de una biblioteca muy popular, con mucha
			comunidad alrededor y muchos años de madurez
		\item Las interfaces se construyen mediante la anidación de
			diversos \textbf{widget}. Tiene una enorme variedad de
			widgets (ejecutar ``\texttt{gtk3-widget-factory}'')
		\item Dispone de una documentación detallada y de calidad
			(\url{https://developer.gnome.org/gtk3/stable/}). Además
			de numerosos tutoriales y ejemplos desarrollados por su
			gran comunidad.
	\end{itemize}
\end{framesec}

\begin{framesec}{Bucle principal}
	\begin{itemize}
		\item Toda la lógica de la interfaz se realiza dentro de un
			bucle infinito \lstinline[language=C]|gtk\_main();|
		\item Antes del bucle la interfaz no es funcional
		\item Después del bucle no existe interfaz
		\item No podemos modificar el interior del bucle
		\item \textbf{¿Cómo interacciono con la interfaz gráfica?}
	\end{itemize}
	\lstinputlisting[language=C, basicstyle=\scriptsize]{csrc/e1/main.c}
\end{framesec}

\begin{framesec}{Señales}
	\begin{itemize}
		\item Una señal puede verse como un \textbf{mensaje} que nos
			envía GTK cada vez que ocurre un \textbf{evento}
		\item Un \textbf{evento} puede ser cualquier acción del usuario:
			un click en un botón, mover el ratón sobre una ventana,
			arrastar algún objeto, \ldots. También hay eventos
			internos de la interfaz gráfica: se acaba de
			crear/destruir la ventana, se va a redibujar algún
			objeto, \ldots
		\item En GTK se implementan mediante \textbf{callbacks}
	\end{itemize}
\end{framesec}

\begin{framesec}{Callbacks}
	\begin{itemize}
		\item Un callback es una función que pasamos a alguna librería,
			para que ella se encargue de llamarla cuando convenga
		\item GTK guarda una lista de punteros a funciones por cada
			posible evento
		\item Como usuarios de GTK, \textbf{conectamos} una o más
			funciones de callback a una señal (pulsar un botón).
			Cuando ocurra el evento oportuno, GTK se encargará de
			llamarlas a todas
	\end{itemize}
\end{framesec}

\begin{framesec}{Glade y builder}
	\begin{itemize}
		\item Crear una interfaz compleja directamente en un lenguaje de
			programación es complejo y poco versátil
		\item GTK puede crear e inicializar todos los objetos necesarios
			a partir de un XML con la descripción de la interfaz
		\item \textbf{Glade} es un programa que nos ayuda a construir
			este XML de manera gráfica
	\end{itemize}
	\vspace{1em}
	\ig{img/glade}{}{0.5}
	\note{
		\begin{itemize}
			\item Hacer que instalen y ejecuten glade
			\item Que creen un ``\texttt{build.ui}'' con:
				\texttt{GtkWindow -> GtkGrid -> (GtkButton, GtkButton)}
		\end{itemize}
	}
\end{framesec}

\begin{framesec}{Ejemplo simple}
	\only<1| handout:1>{\lstinputlisting[language=C, basicstyle=\tiny, linerange={1-6,8-28}]{csrc/e2/main.c}}
	\only<2| handout:2>{\lstinputlisting[language=C, basicstyle=\tiny, linerange={29-45}, firstnumber=27]{csrc/e2/main.c}}
	\only<2| handout:2>{\lstinline[mathescape, language=bash]|gcc -g\ main.c\ $\$$(pkg-config --cflags gtk+-3.0 --libs gtk+-3.0)|}

	\note{
		\begin{itemize}
			\item Explicar el código
			\begin{itemize}
				\item Explicar \textbf{builder} y objetos
				\item Explicar \textbf{castings}
				\item Explicar \textbf{señales}
			\end{itemize}

			\item Que copien y ejecuten el ejemplo
		\end{itemize}
	}
\end{framesec}

\begin{framesec}{Juego de la vida}
	\centering \Huge \bfseries Juego de la vida
	\vspace{1em}
	\ig{img/gol}{}{0.8}
\end{framesec}

\begin{framesubsec}{main}
	\lstinputlisting[language=C, basicstyle=\tiny, linerange={13-39}]{csrc/final/main.c}
\end{framesubsec}

\begin{framesubsec}{gui object}
	\lstinputlisting[language=C, basicstyle=\tiny, linerange={11-29}]{csrc/final/gui_huecos.c}
\end{framesubsec}

\begin{framesubsec}{drawing area}
	\lstinputlisting[language=C, basicstyle=\tiny, linerange={67-76}]{csrc/final/gui_huecos.c}
	\lstinputlisting[language=C, basicstyle=\tiny, linerange={109-124}]{csrc/final/gui_huecos.c}
	\note{Tutorial de cairo y gtk (enlazar en la web): \url{http://zetcode.com/gfx/cairo/}}
\end{framesubsec}

\begin{framesubsec}{Timer}
	\lstinputlisting[language=C, basicstyle=\tiny, linerange={142-143, 148-151}]{csrc/final/gui_huecos.c}
	\lstinputlisting[language=C, basicstyle=\tiny, linerange={126-134}]{csrc/final/gui_huecos.c}
\end{framesubsec}

\begin{framesubsec}{Ejercicio}
	\begin{enumerate}
		\item Estudia detenidamente el código proporcionado
		\item Crea el fichero ``\texttt{builder.ui}'' con
			\textbf{Glade}. Presta atención al identificador de cada
			widget
		\item Completa el código de ``\texttt{gui.c}''. Las zonas a
			completar están marcadas con un \texttt{TODO}
	\end{enumerate}
\end{framesubsec}

\end{document}
