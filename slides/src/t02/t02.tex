\documentclass{mybeamer}

\institute{
	{\textsl{\large Tema 2}}
	\\[1em]
	\textbf{\Large Configurando entorno de trabajo}
}

\begin{document}
\begin{frame}
\titlepage
\end{frame}

\begin{frame}
\frametitle{Índice}
\tableofcontents
\end{frame}

\begin{framesec}{Material de clase}
	\textbf{\Large Wiki}

	{\small \url{http://1984.lsi.us.es/wiki-c}}
	\\[2em]
	\textbf{\Large Lista de correo}

	{\small \url{https://listas.us.es/mailman/listinfo/programacion-c}}
\end{framesec}

\begin{framesec}{Linux}
	\begin{minipage}[0.2\textheight]{\textwidth}
	\begin{columns}[c]
		\begin{column}{0.8\textwidth}
		GNU/Linux es el sistema operativo que vamos a utilizar durante
		el curso.

		\begin{itemize}
			\item Ofrece muchísimas facilidades al programador
			\item Software libre y gratuito
			\item Repositorios con infinidad de herramientas a
				nuestra disposición
		\end{itemize}
		\end{column}

		\begin{column}{0.2\textwidth}
			\ig{img/tux}{}{}
		\end{column}
	\end{columns}
	\end{minipage}
\end{framesec}

\begin{framesubsec}{Consola}
	\only<1| handout:1>{
		\centering
		\textbf{CTRL + ALT + T}
		\\[1em]
		\ig{img/terminals}{}{}

		\note{
			\textbf{vimtutor} para los que ya estén familiarizados
			con la consola de linux
		}
	}

	\only<2| handout:2>{
		Comandos básicos:
		\begin{description}[labelwidth=1cm]
			\item [\texttt{ls}:          ] Lista directorios
			\item [\texttt{cd <dir>}:    ] Cambia a directorio
			\item [\texttt{mkdir <dir>}: ] Crea directorio
			\item [\texttt{touch <file>}:] Crea archivo vacío*
			\item [\texttt{rm <file>}:   ] Borra archivo
			\item [\texttt{rm -r <dir>}: ] Borra directorio y lo que
				hay dentro
		\end{description}

		\note{
			* touch realmente modifica la fecha de edición de un
			archivo, pero si no existe lo crea
			\\[1em]
			Ejercicio (o vimtutor):
			\begin{itemize}
				\item crear carpeta1
				\item crear archivo1 dentro de la careta1
				\item crear carpeta2 dentro de la careta1
				\item crear archivo2 dentro de la careta2
				\item crear archivo3 dentro de la careta2
				\item volver HOME
				\item borrar archivo1
				\item borrar archivo2 y archivo3
				\item borrar carpeta1 y carpeta2
			\end{itemize}
		}
	}

\end{framesubsec}

\begin{framesubsec}{Instalando herramientas}
	Comandos para instalar:
	\begin{description}[align=right,labelwidth=10cm]
		\item[\texttt{sudo <comando>}:               ] Ejecuta un
			comando con permisos de administrador
		\item[\texttt{apt-get install <programa>}:   ] Instala un
			programa del repositorio
	\end{description}

	Programas a instalar (\texttt{sudo apt-get install <programa>}):
	\begin{itemize}
		\item \textbf{gcc}: Compilador
		\item \textbf{git}: Gestor de versiones
		\item \textbf{vim}: Editor de texto
		\item \textbf{geany}: Editor de texto gráfico
	\end{itemize}
\end{framesubsec}

\begin{framesec}{Git}
	\begin{columns}[c]
		\column{0.5\textwidth}
			\centering
			\ig{img/git_logo}{}{}
		\column{0.5\textwidth}
			\centering
			\ig{img/tux}{}{}
	\end{columns}
\end{framesec}

\note[itemize]{
	\item Sistema de control de versiones
	\item Nacimiento:
	\begin{itemize}
		\item Retiran la licencia de uso gratuito de BitKeeper
		\item Ningún sistema de control de versiones cumplía los requisitos
			necesarios
		\item Crean el suyo propio:
		\begin{itemize}
			\item toman CVS como ejemplo de lo que NO hacer
			\item se basan el flujo de trabajo de BitKeeper
			\item énfasis en las protecciones contra corrupción de
				datos
		\end{itemize}
	\end{itemize}
}

\begin{framesubsec}{Características}
	\begin{itemize}
		\item Historial de versiones
		\item Visualización de cambios
		\item Revertir cambios
		\item Trabajo en equipo de forma concurrente
		\item Integridad de los archivos
		\item Sistema distribuido
	\end{itemize}
\end{framesubsec}

\begin{framesubsec}{Distribuido VS Centralizado}
	\begin{columns}[t]
		\column{0.4\textwidth}
			\centering
			\textbf{\large Centralizado}
			\\[1em]
			\ig{img/repo_centr}{}{}
		\column{0.1\textwidth}
			\hspace{0.1\textwidth}
		\column{0.4\textwidth}
			\centering
			\textbf{\large Distribuido}
			\\[1em]
			\ig{img/repo_distr}{}{}
	\end{columns}
\end{framesubsec}

\begin{framesubsec}{Funcionamiento}
	\centering
	\ig{img/git_scheme}{}{}
	\\[2em]
	\begin{itemize}
		\item Instantáneas del estado del repo
		\item Un comentario por cada instantánea
		\item Solo se guardan las diferencias
		\item Máquina de el tiempo
	\end{itemize}
\end{framesubsec}

\begin{framesubsec}{Cuenta en GitHub}
	\only<0| handout:0>{
		\centering
		\ig{img/github_logo}{}{}
		\vspace{1em}
		\hrule
		\vspace{1em}
		\url{https://github.com/}
	}
	\only<1| handout:1>{
		\centering
		Elegimos un nombre de usuario, una contraseña e introducimos
		nuestro correo
		\vspace{1em}
		\hrule
		\vspace{1em}
		\ig{img/gh_ss/1}{}{0.6}
	}
	\only<2| handout:2>{
		\centering
		Nos aseguramos de que el plan gratuito está seleccionado y
		hacemos click en ``\textit{Finish sing up}''
		\vspace{1em}
		\hrule
		\vspace{1em}
		\ig{img/gh_ss/2}{}{0.7}
	}
	\only<3| handout:3>{
		\centering
		Debemos confirmar la dirección de correo
		\vspace{1em}
		\hrule
		\vspace{1em}
		\ig{img/gh_ss/3}{}{}
	}
	\only<4| handout:4>{
		\centering
		Buscamos el correo de confirmación en nuestro buzón y hacemos
		click en ``\textsl{Verify email address}''
		\vspace{1em}
		\hrule
		\vspace{1em}
		\ig{img/gh_ss/4}{}{0.7}
	}
	\only<5| handout:5>{
		\centering
		¡Ya tenemos cuenta en GitHub! Ahora tenemos que crear un nuevo
		repositorio haciendo click donde apunta la flecha roja
		\vspace{1em}
		\hrule
		\vspace{1em}
		\ig{img/gh_ss/5}{}{}
	}
	\only<6| handout:6>{
		\centering
		El nombre del repositorio tiene que tener el formato
		``\textbf{cursoc-\textless tu\_nombre\_compuesto\textgreater }''
		\vspace{1em}
		\hrule
		\vspace{1em}
		\ig{img/gh_ss/6}{}{0.7}
	}
	\only<7| handout:7>{
		\centering
		Copiamos la dirección HTTPS de nuestro repositorio
		\vspace{1em}
		\hrule
		\vspace{1em}
		\ig{img/gh_ss/7}{}{0.9}
	}
\end{framesubsec}

\section{Flujo de trabajo}
\begin{framesubsec}[Repositorio local]{Repositorio local y README}
	Seguimos paso por paso las instrucciones que nos ofrece GitHub:
	\begin{itemize}
		\item \texttt{mkdir \textless nombre de mi repo\textgreater}
		\item \texttt{vim README.md}
		\item \texttt{git init}
		\item \texttt{git add README.md}
		\item \texttt{git commit -m "primer commit"}
		\item \texttt{git remote add origin {\color{blue} https://github.com/4lice/cursoc-alice.git}}
		\item \texttt{git push -u origin master}
	\end{itemize}
\end{framesubsec}

\begin{framesubsec}{Primer código}
	\centering
	\texttt{\$\textgreater\ vim helloworld.c}
	\\[1em]
	\lstinputlisting[language=C]{csrc/helloworld.c}
	\vspace{1em}
	Compilación y ejecución:
	\begin{itemize}
		\item \texttt{gcc \textless mi\_prog.c\textgreater -o \textless
			mi\_exe\textgreater}
		\item \texttt{./mi\_exe}
	\end{itemize}
\end{framesubsec}

\begin{framesubsec}{Subiendo hola mundo}
	\begin{itemize}
		\item \texttt{git status}
		\item \texttt{git add helloworld.c}
		\item \texttt{git commit -m "hola mundo!"}
		\item \texttt{git push -u origin master}
	\end{itemize}
\end{framesubsec}

\begin{framesubsec}{Editando}
	\lstinputlisting[language=diff]{csrc/hw_patch.diff}
\end{framesubsec}

\begin{framesubsec}{Subiendo cambios}
	\begin{itemize}
		\item \texttt{git status}
		\item \texttt{git add helloworld.c}
		\item \texttt{git commit -m "hola yo!"}
		\item \texttt{git push}
	\end{itemize}
\end{framesubsec}

\end{document}
