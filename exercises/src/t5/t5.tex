\documentclass{tareas}

\Tema{5}

\begin{document}
\maketitle
%TODO: doble página
\newpage
% \phantomsection
% \addcontentsline{toc}{section}{Índice}
\tableofcontents
\phantomsection
\newpage

\section{Cabeceras y funciones}

\begin{problem}[Swap]
\begin{statement}
	Crea la función ``\texttt{swap}'' que permita intercambiar dos enteros,
	ie: el entero 1 toma el valor del entero 2 y viceversa.
\end{statement}
\end{problem}

\begin{problem}[Complejos]
\begin{statement}
	Basándote en el ejemplo de ``person'', implementa una pequeña librería
	para trabajar con números complejos (\texttt{cmplx.h y cmplx.c}).

	\subp Crea la estructura ``\texttt{cmplx}'' con dos campos
	\texttt{double}: ``\texttt{real}'' e ``\texttt{img}''
	\subp Creas las funciones:
	\begin{lstlisting}[frame=none, numbers=none, language=C]
		struct cmplx cmplx_sum(const struct cmplx *a,
		                       const struct cmplx *b);
		struct cmplx cmplx_sub(const struct cmplx *a,
		                       const struct cmplx *b);
		double cmplx_abs(const struct cmplx *c)
		void cmplx_norm(struct cmplx *c);
		void cmplx_print(const struct cmplx *c);
	\end{lstlisting}
	\subp Crea un fichero \texttt{main.c} donde hagas uso de esta librería
\end{statement}
\end{problem}

\section{Compilación por bloques y makefile}
\begin{problem}[Compilación por bloques]
\begin{statement}
	\subp Compila el programa anterior por bloques, creando los módulos
	objeto (*.o).
	\subp Prueba a hacer un cambio en alguna función de la librería, crear
	de nuevo ``\texttt{cmplx.o}'' y compilar sin regenerar
	``\texttt{main.o}''
\end{statement}
\end{problem}

\begin{problem}[Compilación por bloques]
\begin{statement}
	Crea un fichero ``makefile'' que automatice la compilación por bloques.
\end{statement}
\end{problem}

\end{document}
