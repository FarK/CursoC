\documentclass{mybeamer}

\institute{
	{\textsl{\large Tema 3}}
	\\[1em]
	\textbf{\Large Introducción a C}
}

\begin{document}
\begin{frame}
\titlepage
\end{frame}

\begin{frame}
\frametitle{Índice}
\begin{multicols}{2}
	\tableofcontents
\end{multicols}
\end{frame}

\begin{framesec}[Variables]{Variables y tipos}
	\only<1>{
	\begin{itemize}
		\item Variables como \textbf{zona de memoria} reservada de
			\textbf{tamaño específico}
		\item Los tipos:
		\begin{itemize}
			\item Definen el tamaño
			\item Dan una idea del uso que se le van a dar a los
				datos guardados
		\end{itemize}
	\end{itemize}
	}
	\only<beamer| beamer:2>{
		\ig{img/var_int}{}{0.4}
	}
	\only<beamer| beamer:3>{
		\ig{img/var_char}{}{0.4}
	}
	\only<beamer| beamer:4>{
		\ig{img/var_char_cnt}{}{0.4}
	}
	\only<handout>{
		\centering
		\ig{img/var_int}{}{0.2}\hspace{1em}
		\ig{img/var_char}{}{0.2}\\[1em]
		\ig{img/var_char_cnt}{}{0.2}
	}
\end{framesec}

\def\fpf[#1]{(\lstinline[language=C]!\"\%#1\" !)}
\begin{framesec}{Tipos básicos}
	\begin{columns}[onlytextwidth]
	\column{0.5\textwidth}
		\textbf{Tipos:}
	\column{0.5\textwidth}
		\textbf{Modificadores:}
	\end{columns}

	\begin{columns}[onlytextwidth]
	\begin{column}{0.5\textwidth}
		\begin{itemize}
			\item char \fpf[c]
			\item int \fpf[i] ó \fpf[d]
			\item float \fpf[f]
			\item double \fpf[f]
			\item bool
		\end{itemize}
	\end{column}
	\note{hay más tipos: números complejos, punteros, atómico\ldots}

	\begin{column}{0.5\textwidth}
		\begin{itemize}
			\item signed \fpf[hh$\color{myorange}\square$]
			\item unsigned \fpf[u]
			\item short \fpf[h$\color{myorange}\square$]
			\item long \fpf[l$\color{myorange}\square$]
			\item long long \fpf[ll$\color{myorange}\square$]
		\end{itemize}
	\end{column}
	\end{columns}
	\vspace{2em}
	\centering
	Más info sobre formato de printf: \url{http://www.cplusplus.com/reference/cstdio/printf}
\end{framesec}
\let\fpf\undefined

\begin{framesec}{Tipos de tamaño fijo}
	\texttt{\large \textcolor{blue}{\#include} <stdint.h>}
	\\
	\begin{center}
	\texttt{\Large [u]int\_<size>\_t}
	\end{center}

	\begin{columns}[onlytextwidth]
	\begin{column}{0.5\textwidth}
		\begin{itemize}
			\item \texttt{int8\_t}
			\item \texttt{int16\_t}
			\item \texttt{int32\_t}
			\item \texttt{int64\_t}
		\end{itemize}
	\end{column}

	\begin{column}{0.5\textwidth}
		\begin{itemize}
			\item \texttt{uint8\_t}
			\item \texttt{uint16\_t}
			\item \texttt{uint32\_t}
			\item \texttt{uint64\_t}
		\end{itemize}
	\end{column}
	\end{columns}
\end{framesec}

\section{Arrays}
\begin{framesubsec}{Descripción}
	\textbf{\Large Arrays}\\
	\lstinline[language=C]|int array[5] = \{1, 2, 3, 4, 5\};|

	\begin{itemize}
		\item Reserva de memoria \textbf{continua} de forma
			\textbf{estática}
		\item Usos:
		\begin{itemize}
			\item Vector de elementos
			\item Matrices (multidimensionales)
			\item Cadenas de texto
			\item Espacio de memoria (buffer)
		\end{itemize}
	\end{itemize}
\end{framesubsec}

\begin{framesubsec}{Ejemplo}
	\lstinputlisting[language=C]{csrc/array_example.c}
\end{framesubsec}

\begin{framesubsec}{Cadenas}
	\lstinputlisting[language=C]{csrc/array_strings.c}
\end{framesubsec}

\begin{framesubsec}{Array multidimensional}
	\ig{img/array_md}{}{}
\end{framesubsec}

\ejframe[subsec]{tipos, arrays y cadenas}

\begin{framesec}{Punteros}
	\only<1-3| handout:1>{
	\setbeamercovered{transparent}
	\begin{columns}[c]
	\begin{column}{0.4\textwidth}
		\begin{itemize}
			\item<1-3> Son \textbf{variables normales} y corrientes
			\item<2-3> Pensadas para guardar una \textbf{dirección de
				memoria}
			\item<3-3> El tipo del puntero hace referencia al tipo de
				dato \textbf{al que apunta}
		\end{itemize}
	\end{column}

	\begin{column}{0.6\textwidth}
		\only<beamer| 1>{\ig{img/punteros_1}{}{}}
		\only<beamer| 2>{\ig{img/punteros_2}{}{}}
		\only<3>{\ig{img/punteros_3}{}{}}
	\end{column}
	\end{columns}
	}

	\only<4| handout:2>{
		\ig{img/var_ptr}{}{}
	}
\end{framesec}

\begin{framesubsec}{Ejemplo}
	\lstinputlisting[language=C]{csrc/punteros.c}
	\note{
		\begin{itemize}
			\item Que copien, compilen y ejecuten el ejemplo
				y mientras...
			\item Pintar esto en la pizarra: \ig{img/punteros_sintaxis}{}{}
		\end{itemize}
	}
\end{framesubsec}

\begin{framesubsec}{Sintaxis}
	\ig{img/punteros_sintaxis}{}{}
\end{framesubsec}

\begin{framesec}{Arrays y punteros}
	\note{
		Ir poniendo ejemplos:\\
	}

	\setbeamercovered{transparent}
	\begin{itemize}
		\item<1-> \textbf{Arrays}:
		\begin{itemize}
			\item<2-> Son prácticamente punteros constantes (no se
				puede modificar la dirección a la que apunta)
			\item<2-> Apuntan al primer elemento del array
			\note<2>{
				\lstinline[language=C]|int array1[10], array2[10];|\\
				\lstinline[language=C]|// array1 = array2; // Error|\\
				\lstinline[language=C]|printf("\%p == \%p\\n", array1, \&array1[0]);|\\
			}
			\item<3-> Mediante el tipo y el índice se obtiene la
				dirección del elemento deseado
			\note<3>{dibujo de array}
		\end{itemize}

		\item<1-> \textbf{Punteros}:
		\begin{itemize}
			\item<4-> Soportan las operaciones de suma y resta de
				enteros
			\item<4-> Al sumar un entero y un puntero estamos
				sumando a la dirección de memoria ese entero por
				el tamaño del tipo del puntero
			\note<4>{
				\lstinline[language=C]|int *p = \&i;|\\
				\lstinline[language=C]|p + 2; // Esta sumando sizeof(int)*2|
			}

			\item<5-> Se pueden indexar como un array
			\note<5>{
				\lstinline[language=C]|int *p = array;|\\
				\lstinline[language=C]|p[3];|
			}
		\end{itemize}
	\end{itemize}
\end{framesec}

\begin{framesubsec}{Ejemplo}
	\textbf{\Large Ejemplo:}\\
	\lstinputlisting[language=C]{csrc/arrays_punteros.c}

	\note{preguntar qué imprimirá}
\end{framesubsec}

\begin{framesubsec}[Recorriendo arrays]{Formas de recorrer un array}
	\lstinputlisting[language=C]{csrc/arrays_recorriendo.c}
\end{framesubsec}

\ejframe[subsec]{punteros}

\begin{framesec}{Jugando con Punteros}
	\lstinputlisting[language=C, linerange={1-4,6-11}]{csrc/punteros_jugando.c}
	\note{\lstinputlisting[language=C]{csrc/punteros_jugando.c}}
\end{framesec}

\end{document}
