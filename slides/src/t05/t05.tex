\documentclass{mybeamer}

\institute{
	{\textsl{\large Tema 5}}
	\\[1em]
	\textbf{\Large C modular}
}

\begin{document}
\begin{frame}
\titlepage
\end{frame}

\begin{frame}
\frametitle{Índice}
\begin{multicols}{2}
	\tableofcontents
\end{multicols}
\end{frame}

\begin{framesec}{Funciones}
	\begin{columns}[onlytextwidth]
	\begin{column}{0.6\textwidth}
		En C las funciones:
		\begin{itemize}
			\item Retornan un solo valor o nada (\textit{void})
			\item De cero a N parámetros
			\item Cada parámetro es de un tipo específico
			\item Todos los parámetros se pasan \textbf{por copia}
			\item Tienen una \textbf{declaración} y una
				\textbf{definición}
			\item Una función ha de estar declarada antes de ser
				llamada
			\item Una función no tiene por que estar definida a la
				hora de ser llamada
		\end{itemize}
	\end{column}

	\begin{column}{0.4\textwidth}
		\lstinputlisting[numbers=none, language=C, basicstyle=\tiny]{csrc/function_example.c}
	\end{column}
	\end{columns}
\end{framesec}

\subsection[Parámetros]{Paso de parámetros}
\begin{framesubsubsec}{Paso por copia}
	\begin{columns}[onlytextwidth]
	\begin{column}{0.5\textwidth}
		\begin{itemize}
			\item \textbf{Siempre se copia} el parámetro (variable o
				constante) que se le pasa a la función al
				llamarla
			\item Dentro de la función se trabaja con la copia
			\item Las variables originales no se ven afectadas
		\end{itemize}
	\end{column}

	\begin{column}{0.4\textwidth}
		\lstinputlisting[language=C, basicstyle=\scriptsize]{csrc/function_copy.c}
	\end{column}
	\end{columns}
\end{framesubsubsec}

\begin{framesubsubsec}{Paso por referencia}
	\begin{columns}[onlytextwidth]
	\begin{column}{0.5\textwidth}
		\begin{itemize}
			\item Para poder modificar las variables originales
				dentro de una función, esta ha de trabajar con
				la
				\textbf{referencia} a la variable, no con la
				original
			\item Esto se consigue con \textbf{punteros}
		\end{itemize}
	\end{column}

	\begin{column}{0.4\textwidth}
		\lstinputlisting[language=C, basicstyle=\scriptsize]{csrc/function_ref.c}
	\end{column}
	\end{columns}
\end{framesubsubsec}

\begin{framesec}{Cabeceras}
	\begin{itemize}
		\item Podemos crear nuestros propios ficheros \textit{*.h}
		\item Un fichero de cabecera suele tener únicamente
			declaraciones y no definiciones
		\item En el fichero de código (\textit{*.c}) se definen las
			funciones declaradas en la cabecera
		\item Podemos tener una cabecera y varios tipos de definiciones
		\item Nos permite crear una interfaz que aisle al usuario de la
			definición de las funciones
	\end{itemize}
\end{framesec}

\begin{framesubsec}{Ejemplo}
		\textbf{person.h}
		\lstinputlisting[language=C, basicstyle=\tiny]{csrc/person.h}

		\textbf{person.c}
		\lstinputlisting[language=C, basicstyle=\tiny]{csrc/person.c}

		\note{
			explicar paso por referencia para optimizar la copia de
			parámetros y \textbf{const}
		}
\end{framesubsec}


\def\tFiveCompSol{
	\textbf{Solución:}
	\begin{itemize}
		\item Cada fichero de código puede compilarse de manera
			independiente, creando un \textbf{fichero
			objeto} (\textit{*.o})
		\item El \textbf{linker} se encarga de enlazar todos los
			ficheros objetos para crear el ejecutable
		\item Un módulo objeto puede tener referencias a
			símbolos definidos en otro módulo
	\end{itemize}
	\vspace{1em}
	Cómo se genera: \lstinline[language=sh]|gcc -c persona.c|
}

\ejframe{Funciones y cabeceras}

\begin{framesec}[Compilación]{Compilación por bloques}
	\only<1| handout:1>{
	\textbf{Razones:}
	\begin{itemize}
		\item Tiempo:
		\begin{itemize}
			\item La compilación es un proceso complejo y costoso
			\item Proyectos muy grandes pueden tardar mucho tiempo en
				compilar
			\item Cuando se está desarrollando esto se vuelve prohibitivo
		\end{itemize}
		\item Organización:
		\begin{itemize}
			\item Dividir el código en bloques lógicos es una buena
				práctica
			\item Mejora:
			\begin{itemize}
				\item El mantenimiento
				\item La legibilidad
				\item La portabilidad
				\item La escalabilidad
				\item etc
			\end{itemize}
		\end{itemize}
	\end{itemize}
	}

	\only<2| handout:2>{
		\centering
		\ig{img/compilacion}{}{}
		\note{\tFiveCompSol}
	}

	\only<3| handout:3>{
		\tFiveCompSol
	}
\end{framesec}

\begin{framesec}[Make]{Make y Makefile}
	Herramienta para automatizar la compilación del código (y mucho más)
	\begin{itemize}
		\item Gestiona dependencias para no compilar innecesariamente
		\item Facilita enormemente el trabajo
		\item También se suele utilizar para instalación y
			desinstalación
	\end{itemize}
\end{framesec}

\begin{framesubsec}{Estructura}
	\lstinputlisting[language=make, frame=none, numbers=none]{csrc/make_obj}
	\vspace{1em}
	\begin{itemize}
		\item \textbf{Objetivos}: Un objetivo suele ser un archivo que
			se desea generar (por ejemplo, un ejecutable)
		\item \textbf{Prerequisitos}: Lista de objetivos
		\item \textbf{Órdenes}: Instrucciones a realizar para generar el
			objetivo. Siempre van precedidas de una tabulación
	\end{itemize}
\end{framesubsec}

\begin{framesubsec}{Ejemplo}
	\textbf{makefile}
	\lstinputlisting[language=make]{csrc/make_example}
\end{framesubsec}

\ejframe[subsec]{Makefile}

\end{document}
