\documentclass{mybeamer}
\usepackage{beramono} % For bold texttt

\usepackage{tikz}
\newcommand{\tikzmark}[2]{%
	     \tikz[overlay,remember picture] \node[text=black, inner sep=2pt] (#1) {#2};
}

\institute{
	{\textsl{\large Tema 9}}
	\\[1em]
	\textbf{\Large Listas encadenadas}
}

\begin{document}
\begin{frame}
\titlepage
\end{frame}

\begin{frame}
\frametitle{Índice}
% \begin{multicols}{2}
	\tableofcontents
% \end{multicols}
\end{frame}

\begin{framesec}[¿Qué es?]{¿Qué es una lista encadenada?}
	\ig{img/list}{}{}
	\vspace{2em}
	\begin{itemize}
		\item Cada elemento guarda:
		\begin{enumerate}
			\item Información
			\item Una referencia al siguiente elemento [y al
				anterior]
		\end{enumerate}
		\item Los elementos están dispersos en la memoria. Se reservan
			individualmente.
	\end{itemize}
\end{framesec}

\begin{framesec}[Utilidad]{¿Cuándo son útiles?}
	\begin{itemize}
		\item \textbf{No sabemos cuántos} elementos vamos a tener que
			guardar
		\item Vamos a recorrer los elementos de manera
			\textbf{secuencial}
		\item Necesitamos hacer \textbf{inserciones} y/o
			\textbf{eliminaciones} de elementos o sublistas
	\end{itemize}
\end{framesec}

\section{Operaciones}
\begin{framesubsec}{Insertar}
	\centering
	\ig{img/list_ins1.png}{}{0.4}
	\vspace{1em}
	\ig{img/list_ins2.png}{}{0.4}
\end{framesubsec}

\begin{framesubsec}{Eliminar}
	\centering
	\ig{img/list_del1.png}{}{0.4}
	\vspace{1em}
	\ig{img/list_del2.png}{}{0.4}
\end{framesubsec}

\begin{framesec}{Lista del Kernel}
	\begin{columns}[t]
		\begin{column}{0.5\textwidth}
			\ig{img/list_head}{}{}
			\vspace{1.1em}
			\lstinputlisting[language=C, numbers=none, frame=none]{csrc/list_head.c}
		\end{column}
		\hspace{1em}
		\begin{column}{0.5\textwidth}
			\ig{img/list_str}{}{}
			\vspace{1em}
			\lstinputlisting[language=C, numbers=none, frame=none]{csrc/list_str.c}
		\end{column}
	\end{columns}
\end{framesec}

\begin{framesubsec}{Ejemplo}
	\setbeamercovered{transparent}
	\only<1| handout:1>{\ig{img/list_1}{}{}\par}
	\only<2| handout:2>{\ig{img/list_2}{}{}\par}
	\only<3| handout:3>{\ig{img/list_3}{}{}\par}
	\only<4| handout:4>{
		\lstinputlisting[language=bash, numbers=none, frame=none, xleftmargin=0.35\textwidth, linewidth=0.65\textwidth]{csrc/list.out}
		\vspace{1em}
	}
	\lstinputlisting[language=C, basicstyle=\tiny, linerange={13-31}, escapeinside=@@]{csrc/list.c}
\end{framesubsec}

\begin{framesubsec}[list\_entry]{¿Cómo se obtiene el elemento a partir del \texttt{list\_head}?}
	\centering
	\ig{img/list_entry}{}{}
	\vspace{1em}
	\lstinputlisting[language=C, basicstyle=\tiny, linerange={183-190}]{csrc/list.h}
\end{framesubsec}

\begin{framesubsec}[Funciones]{Funciones principales}
	\begin{itemize}
		\item \texttt{\bfseries list\_add}: Añade \textbf{después} de la cabeza
			(\textbf{pila})
		\item \texttt{\bfseries list\_add\_tail}: Añade \textbf{antes} de la
			cabeza (\textbf{cola})
		\item \texttt{\bfseries list\_move}: Mueve un elemento de una lista a
			\textbf{después} de la cabeza de otra (\textbf{pila})
		\item \texttt{\bfseries list\_move\_tail}: Mueve un elemento de una lista a
			\textbf{antes} de la cabeza de otra (\textbf{cola})
		\item \texttt{\bfseries list\_del}: Borra el nodo que recibe
		\item \texttt{\bfseries list\_splice}: Une dos listas
		\item \texttt{\bfseries list\_empty}: Comprueba si una lista está vacía
	\end{itemize}
\end{framesubsec}

\begin{framesubsec}[Macros]{Macros principales}
	\begin{itemize}
		\item \texttt{\bfseries INIT\_LIST\_HEAD}: Inicializa la cabeza
			de una lista
		\item \texttt{\bfseries list\_for\_each}: Recorre cada
			\textbf{nodo} de una lista
		\item \texttt{\bfseries list\_for\_each\_safe}: Recorre cada
			\textbf{nodo} de una lista y se puede borrar un
			elemento mientras se rocorre la lista.
		\item \texttt{\bfseries list\_for\_each\_entry}: Recorre cada
			\textbf{elemento} de una lista
		\item \texttt{\bfseries list\_for\_each\_entry\_safe}: Recorre cada
			\textbf{elemento} de una lista y se puede borrar un
			elemento mientras se rocorre la lista.
	\end{itemize}
\end{framesubsec}

\end{document}
