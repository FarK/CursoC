\documentclass{mybeamer}

\institute{
	{\textsl{\large Anexo}}
	\\[1em]
	\textbf{\Large Como usar listas encadenadas en el Juego de la Vida}
}

\begin{document}

\begin{frame}
\titlepage
\end{frame}

\begin{frame}
\frametitle{Índice}
\tableofcontents
\end{frame}

\begin{framesec}{Introducción}
	\begin{itemize}
		\item Las únicas células que pueden (o no) cambiar de estados
			son \textbf{las vivas y sus vecinas}
		\item Para mundos grandes con pocas células se comprueban muchas
			células muertas inútilmente
		\item Una lista de células vivas nos permite recorrer las
			células de interés, obviando todas las que no pueden
			cambiar de estado
		\item Esto puede mejorar el rendimiento en mundos grandes con
			pocas células, pero podría empeorarlo en mundos pequeños
			con muchas células en un estado estable
	\end{itemize}
\end{framesec}

\begin{framesec}{Estructura}
	Utilizaremos tres listas cuyos nodos guardarán las coordenadas de una
	célula de nuestro array:
	\begin{itemize}
		\item \textbf{alive\_cells}: Lista de células vivas
		\item \textbf{to\_kill}: Lista de células que mueren en
			la siguiente iteración
		\item \textbf{to\_revive}: Lista de células que
			nacen/reviven en la siguiente iteración
	\end{itemize}
\end{framesec}

\begin{framesec}{Algoritmo}
	\begin{enumerate}
		\item Recorremos \textbf{alive\_cells}, y por cada nodo:
		\begin{itemize}
			\item Comprobamos si sobrevive:
			\begin{itemize}
				\item SÍ: no hacemos nada
				\item NO: movemos el nodo a
					\textbf{to\_kill}
			\end{itemize}
			\item Comprobamos si nace alguna de sus vecinas
				muertas
			\begin{itemize}
				\item SÍ: creamos un nuevo nodo en
					\textbf{to\_revive}
				\item NO: no hacemos nada
			\end{itemize}
		\end{itemize}

	\item Recorremos \textbf{to\_kill}, y por cada nodo:
			cambiamos el estado a \texttt{MUERTA} en el array y
			eliminamos el nodo

		\item Recorremos \textbf{to\_revive}, y por cada nodo:
		\begin{itemize}
			\item Si es una célula muerta: Revivir y mover el nodo a
				\textbf{alive\_cells}
			\item Si es una célula viva: Eliminar el nodo
		\end{itemize}
	\end{enumerate}
\end{framesec}

\newcounter{paso}
\setcounter{paso}{1}
\def\pasofw{\ifnum\value{paso}<10 0\fi\arabic{paso}}

\newcommand{\paso}[1]{
	\begin{framesubsec}{Paso \arabic{paso}}
		#1

		\vspace{0.5em}
		\begin{center}
			\ig{img/example\pasofw}{}{}
		\end{center}
	\end{framesubsec}
	\stepcounter{paso}
}

%1%
\paso{
	Las células negras son las vivas y junto sus vecinas grises, son las
	únicas que pueden cambiar de estado
}

%2%
\paso{
	La célula muere, movemos el nodo a \textit{to\_kill} y añadimos los dos
	nacimientos a \textit{to\_revive}
}

%3%
\paso{
	La célula se queda viva. Añadimos de nuevo otros dos nodos a
	\textit{to\_revive} (no importa que estén repetidos)
}

%4%
\paso{
	La célula muere, movemos el nodo a \textit{to\_kill} y añadimos de nuevo
	dos nodos a \textit{to\_revive}
}

%5%
\paso{
	Marcamos la célula como muerta en el array y eliminamos el nodo
}

%6%
\paso{
	Marcamos la célula como muerta en el array y eliminamos el nodo
}

%7%
\paso{
	La célula está muerta así que movemos el nodo a \textit{alive\_cells} y
	la marcamos como viva en el array
}

%8%
\paso{
	La célula está muerta así que movemos el nodo a \textit{alive\_cells} y
	la marcamos como viva en el array
}

%9%
\paso{
	La célula ya está viva así que eliminamos el nodo
}

%10%
\paso{
	La célula ya está viva así que eliminamos el nodo
}

%11%
\paso{
	Esto mismo ocurre con todos los nodos ya revividos (no importa que estén
	duplicados)
}

%12%
\paso{
	Ya tenemos la iteración completada y nuestra lista \textit{alive\_cells}
	preparada para la siguiente
}

\end{document}
