\documentclass{mybeamer}
\usepackage{marvosym}

\institute{
	{\textsl{\large Tema 15}}
	\\[1em]
	\textbf{\Large Sockets}
}

\begin{document}
\begin{frame}
\titlepage
\end{frame}

\begin{frame}
\frametitle{Índice}
% \begin{multicols}{2}
	\tableofcontents
% \end{multicols}
\end{frame}

\begin{framesec}[redes]{Un poco de redes}
	\begin{itemize}
		\item \textbf{IP}: Dirección de cada máquina. Por ejemplo: 192.168.0.1
		\item \textbf{TCP}: Protocolo \textbf{con conexión}. Garantiza
			la llegada de los paquetes en orden
		\item \textbf{UDP}: Protocolo \textbf{sin conexión}. No
			garantiza ni la llegada, ni el orden de los paquetes;
			pero si llegan, llegan sin errores
		\item \textbf{Puerto}: Proceso/servicio/cliente de cada máquina
	\end{itemize}
\end{framesec}

\begin{framesec}{Conexión TCP}
	\centering \Large \textit{Apretón de manos en tres pasos}\\
	(\textit{3-way handshake})
	\\[2em]
	\ig{img/handshake}{}{}
\end{framesec}

\begin{framesec}{UDP}
	\centering \Large Sin conexión
	\\[2em]
	\ig{img/udp}{}{}
\end{framesec}

\begin{framesec}{Sockets}
	\begin{itemize}
		\item Un socket es una manera de hablar con otros procesos a
			través de descriptores de ficheros Unix
		\item Estos programas no tienen por qué estar en la misma
			máquina. La comunicación a través de internet se hace de
			forma transparente
		\item Puedes enviar y recibir cualquier información de otro
			proceso en otra máquina, en la otra punta del mundo, tan
			solo escribiendo y leyendo de un fichero
	\end{itemize}

	\vspace{1em}
	\centering ``\textit{En Unix todo es un fichero}''
\end{framesec}

\begin{framesec}{Byte order}
	``\textit{There really is no easy way to say this, so I'll just blurt it
	out: your computer might have been storing bytes in reverse order behind
	your back. I know! No one wanted to have to tell you.}''
	\hspace*\fill--- Beej Jorgensen \href{http://www.beej.us/guide/bgnet/output/html/singlepage/bgnet.html}{Guide to Network Programming}
	\vspace{1em}

	\begin{itemize}
		\item No todas las máquinas almacenan la información en memoria
			en el mismo orden
		\item La comunicación entre máquinas con distinto
			\textit{endianess} es un problema
		\item El programador tiene que ser consciente de esto al enviar
			y recibir información de una máquina remota
		\item Las siguientes funciones nos facilitan la tarea:
		\begin{itemize}
			\item \lstinline[language=C]|uint32\_t htonl(uint32\_t hostlong)|
			\item \lstinline[language=C]|uint16\_t htons(uint16\_t hostshort)|
			\item \lstinline[language=C]|uint32\_t ntohl(uint32\_t netlong)|
			\item \lstinline[language=C]|uint16\_t ntohs(uint16\_t netshort)|
		\end{itemize}
	\end{itemize}

	\note{Explicar \textit{Hardware-TO-Network} y \textit{Network-TO-Hardware}}
\end{framesec}

\begin{framesec}[Estructuras]{Estructuras para direcciones}
	\only<1| handout:1>{
	Por temas de herencia histórica y retrocompatibilidad, las estructuras
	para representar una dirección de red son un poco caóticas:
	\vspace{1em}

	\lstinline[language=C]|struct addrinfo|:
		Información sobre direccion, socket, protocolo, \ldots (es una lista)

	\lstinline[language=C]|struct sockaddr\_storage|:
		Estructura genérica para direcciones (cabe IPv6)
	\begin{itemize}
		\item \lstinline[language=C]|struct sockaddr|:
			Estructura genérica para direcciones (no cabe IPv6)
		\begin{itemize}
			\item \lstinline[language=C]|struct sockaddr\_in|:
				Estructura para direcciones IPv4. (puerto + dirección)
			\begin{itemize}
				\item \lstinline[language=C]|struct in\_addr|:
					Estructura para direcciones IPv4
			\end{itemize}
			\item \lstinline[language=C]|struct sockaddr\_in6|:
				Estructura para direcciones IPv6. (puerto + dirección + info extra)
			\begin{itemize}
				\item \lstinline[language=C]|struct in6\_addr|:
					Estructura para direcciones IPv6
			\end{itemize}
		\end{itemize}
	\end{itemize}
	}

	\only<2| handout:2>{\lstinputlisting[language=C, basicstyle=\tiny, linerange={1-28}]{csrc/structs.c}}
	\only<3| handout:3>{\lstinputlisting[language=C, basicstyle=\tiny, linerange={30-40}]{csrc/structs.c}}
\end{framesec}

\begin{framesec}{Direcciones y cadenas}
	Esta par de funciones nos permiten convertir entre el número binario de
	una dirección y su representación como cadena de texto:
	\vspace{1em}

	\begin{itemize}
		\item \lstinline[language=C]|const char *inet_ntop(int af, const void *src, char *dst, socklen_t size)|
		\item \lstinline[language=C]|int inet_pton(int af, const char *src, void *dst)|
	\end{itemize}

	Mediante el parámetro \lstinline[language=C]|int af|, indicamos el tipo
	de dirección que estamos tratando: (\lstinline[language=C]|AF_INET| ó
	\lstinline[language=C]|AF_INET6|)
	\\[1em]
	Estas funciones están obsoletas por no tratar bien con IPv6, por lo que
	se desaconseja su uso:
	\begin{itemize}
		\item \lstinline[language=C]|gethostbyname()|
		\item \lstinline[language=C]|gethostbyaddr()|
	\end{itemize}
\end{framesec}

\begin{framesec}{Estructura cliente-servidor}
	\only<1| handout:1>{
	\centering \textbf{\Large Funciones principales}
	\begin{table}[]
	\centering
	\begin{tabular}{|l|l|l|l|}
	\hline
	\multicolumn{2}{|c|}{\textbf{TCP}} & \multicolumn{2}{c|}{\textbf{UDP}}     \\ \hline
	\textbf{Server}  & \textbf{Client}   & \textbf{Server}  & \textbf{Client}  \\ \hline
	socket()         & socket()          & socket()         & socket()         \\ \hline
	bind()           &                   & bind()           &                  \\ \hline
	listen()         &                   &                  &                  \\ \hline
	accept()         & connect()         &                  &                  \\ \hline
	recv()           & recv()            & recvfrom()       & recvfrom()       \\
	send()           & send()            & sendto()         & sendto()         \\ \hline
	close()          & close()           & close()          & close()          \\ \hline
	\end{tabular}
	\end{table}
	}

	\only<2| handout:2>{
		\begin{itemize}
			\item \textbf{socket()}: Crea el socket (devuelve un
				descriptor de fichero)
			\item \textbf{bind()}: Asocia una dirección al socket
			\item \textbf{listen()}: Pone el servidor en modo
				escucha para aceptar peticiones de conexión
			\item \textbf{accept()}: Acepta la primera petición de
				la cola o se queda esperando hasta que llegue
				una
			\item \textbf{connect()}: Intenta establecer una
				conexión con un servidor
			\item \textbf{send()/sendto}: Envía información
			\item \textbf{recv()/recvfrom}: Recibe información
			\item \textbf{close()}: Cierra un descriptor de fichero
				(en nuestro caso, el socket)
		\end{itemize}
	}
\end{framesec}

\begin{framesec}{Código}
	\centering \Huge \bfseries Estudiar código cliente-servidor
\end{framesec}

\end{document}
