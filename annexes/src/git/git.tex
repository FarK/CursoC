\documentclass{mybeamer}

\institute{
	{\textsl{\large Anexo}}
	\\[1em]
	\textbf{\Large GIT}
}

\begin{document}
\begin{frame}
\titlepage
\end{frame}

\begin{frame}
\frametitle{Índice}
\tableofcontents
\end{frame}

\begin{framesec}{Cuenta en GitHub}
		\centering
		\href{https://github.com}{github.com}\\
		Elegimos un nombre de usuario, una contraseña e introducimos
		nuestro correo
		\vspace{1em}
		\hrule
		\vspace{1em}
		\ig{img/gh_ss/1}{}{0.6}
\end{framesec}

\begin{framesubsec}{Plan gratuito}
		\centering
		Nos aseguramos de que el plan gratuito está seleccionado y
		hacemos click en ``\textit{Finish sing up}''
		\vspace{1em}
		\hrule
		\vspace{1em}
		\ig{img/gh_ss/2}{}{0.7}
\end{framesubsec}

\begin{framesubsec}{Confirmar correo}
	\only<1>{
		\centering
		Debemos confirmar la dirección de correo
		\vspace{1em}
		\hrule
		\vspace{1em}
		\ig{img/gh_ss/3}{}{}
	}
	\only<2>{
		\centering
		Buscamos el correo de confirmación en nuestro buzón y hacemos
		click en ``\textsl{Verify email address}''
		\vspace{1em}
		\hrule
		\vspace{1em}
		\ig{img/gh_ss/4}{}{0.7}
	}
\end{framesubsec}

\section{Fork de mi repositorio}
\begin{framesubsec}{Buscar mi repositorio}
	\centering
	Entramos en la cuenta del profesor
	(\href{https://github.com/profedotc?tab=repositories}{github.com/profedotc}),
	y en la pestaña ``\textsl{Repositories}'' buscamos el repositorio que
	tenga nuestro nombre y apellido
	\vspace{1em}
	\hrule
	\vspace{1em}
	\ig{img/gh_ss/5}{}{0.6}
\end{framesubsec}

\begin{framesubsec}{Fork}
	\centering
	Hacemos clien en ``\textsl{Fork}'' para crear una copia del repositorio
	en nuestra cuenta
	\vspace{1em}
	\hrule
	\vspace{1em}
	\ig{img/gh_ss/6}{}{0.7}
\end{framesubsec}

\begin{framesubsec}{Clonar mi repositorio}
	\centering
	\only<1>{
		En el menú ``\textsl{Clone or download}'' podemos encontrar la URL
		necesaria para clonar nuestro repositorio
		\vspace{1em}
		\hrule
		\vspace{1em}
		\ig{img/gh_ss/7}{}{0.7}
	}
	\only<2>{
		Para clonar nuestro repositorio abrimos un terminal, navegamos
		hasta la carpeta dónde lo queramos clonar y ejecutamos el
		siguiente comando de git:
		\vspace{3em}\\
		\lstinline[language=bash]|> git clone https://github.com/profedotc/alice_wonderland.git|
	}
\end{framesubsec}

\end{document}
