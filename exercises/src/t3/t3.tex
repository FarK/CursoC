\documentclass{tareas}

\Tema{3}

\begin{document}
\maketitle
%TODO: doble página
\newpage
% \phantomsection
% \addcontentsline{toc}{section}{Índice}
\tableofcontents
\phantomsection
\newpage

\section{Tipos, arrays y cadenas}
\begin{problem}[printf y sizeof]
\begin{statement}
	Crea, inicializa e imprime el valor y el tamaño de al menos 5 variables
	de distinto tipo:
	\\
	\lstinputlisting[language=C]{csrc/sizeof.c}
	\vspace{-2em}
\end{statement}
\end{problem}


\begin{problem}[Arrays]
\begin{statement}
	Crea un array de 10 elementos y juega con él:

	\begin{subproblems}
		\subp Recorre e imprime el array
		\subp Recorrer los elementos de posición par en orden inverso
		\subp Cambia el elemento $i$ por el $i+1$ (y el último por el
		primero)
		\subp Cambia el elemento $i$ por el $i-1$ (y el primero por el
		último)
		\subp Espeja el array: $array[i] = array[N], array[i+1] =
		array[N-1], \ldots$
	\end{subproblems}
\end{statement}
\end{problem}


\begin{problem}[Cadenas]
\begin{statement}
	Crea una cadena cualquiera (con mayúsculas y minúsculas) juega con ella.
	\textbf{Pista}: Consulta la
	\href{https://es.wikipedia.org/wiki/ASCII#Caracteres_imprimibles_ASCII}{tabla
	ASCII} y ten en cuenta la diferencia entre el código de una letra en
	minúsculas y una en mayúsculas.

	\begin{subproblems}
		\subp Convierte la cadena entera a mayúsculas
		\subp Convierte a mayúsculas solo la primera letra de cada
		palabra
	\end{subproblems}
\end{statement}
\end{problem}


\begin{problem}[Matrices]
\begin{statement}
	Crea dos matriz de dos dimensiones (3x3), inicialízalas y juega con
	ellas:

	\begin{subproblems}
		\subp Imprime la matriz
		\subp Multiplica todos los elementos por 2
		\subp Multiplica las dos matrices entre sí
	\end{subproblems}
\end{statement}
\end{problem}


\clearpage
\section{Punteros}
\begin{problem}[Sintáxis]
\begin{statement}
	Crea un pequeño programa que haga lo siguiente:
	\begin{enumerate}
		\item Declara e inicializa un vector de 8 \lstinline[language=C]|char|
		\item Declara un puntero a \lstinline[language=C]|char| y
			asígnale la dirección de memoria del 1$^o$ elemento del
			vector
		\item Suma 4 al puntero anterior
		\item Cambia el valor de la dirección de memoria a la que apunta
			el puntero anterior
		\item Crea un bucle que imprima los 8 elementos del array en
			hexadecimal
		\item Declara un puntero a \lstinline[language=C]|int| y
			asígnale el valor (dirección de memoria) del puntero
			anterior (no olvides el casting
			\lstinline[language=C]|(int *)|
		\item Imprime en hexadecimal el valor al que apunta el puntero a
			\lstinline[language=C]|int|
	\end{enumerate}
	\vspace{-2em}
\end{statement}
\end{problem}

\begin{problem}[Arrays y punteros]
\begin{statement}
	En el mismo programa realiza los siguientes apartados:
	\begin{subproblems}
		\subp Crea e inicializa un vector de enteros y otro de reales de
		5 elementos cada uno
		\subp Crea otro vector de float de 5 elementos sin inicializar
		\subp Declara dos punteros para cada array
		\subp Utiliza la ``\textbf{forma 1}'' para imprimir los dos
		vectores
		\subp Utiliza la ``\textbf{forma 3}'' multiplicar los elementos
		del primer y el segundo vector y guardarlos en el tercero:
		$v3[i] = v1[i] * v2[i]$
		\subp Crea una cadena y dos punteros a char
		\subp Utiliza la ``\textbf{forma 2}'' y los dos punteros para
		invertir la cadena: $str[i] = str[N-i]$
	\end{subproblems}
\end{statement}
\end{problem}

\begin{problem}[Matrices]
\begin{statement}
	En el mismo programa realiza los siguientes apartados:
	\begin{subproblems}
		\subp Crea una matriz de enteros de 3x3 e inicialízala
		\subp Crea un puntero que apunte a la matriz
		\subp Crea dos enteros $i$ y $j$
		\subp Utiliza los enteros y el puntero para recorrer e imprimir
		la matriz la matriz. \textbf{Restricciones:} No puedes utilizar
		el puntero como un array (\lstinline[language=C]|p[i][j]|)
	\end{subproblems}
\end{statement}
\end{problem}

\end{document}
