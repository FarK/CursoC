\documentclass{mybeamer}

\institute{
	{\textsl{\large Tema 4}}
	\\[1em]
	\textbf{\Large Estructuras}
}

\begin{document}
\begin{frame}
\titlepage
\end{frame}

\begin{frame}
\frametitle{Índice}
\begin{multicols}{2}
	\tableofcontents
\end{multicols}
\end{frame}

\begin{framesec}{Struct}
	Lista de variables agrupadas físicamente en un mismo bloque de memoria.
	\lstinputlisting[language=C]{csrc/struct.c}
\end{framesec}

\begin{framesubsec}{Alineación y tamaño}
	\begin{columns}[onlytextwidth]
		\begin{column}{0.5\textwidth}
			\lstinputlisting[numbers=none, frame=none, language=C]{csrc/struct_sizeof.c}
		\end{column}

		\begin{column}{0.5\textwidth}
			\only<2| handout:1>{\ig{img/struct_padding}{}{}}
		\end{column}
	\end{columns}
	\vspace{2em}
	\centering
	\lstinline[language=C]|sizeof(struct ejemplo);|%
	\only<beamer | beamer:1>{¿5 bytes?}
	\only<2| handout:1>{\sout{¿5 bytes?} \textrightarrow 12 bytes}
\end{framesubsec}

\begin{framesubsec}{Anidamiento}
	\lstinputlisting[language=C]{csrc/struct_annidated.c}
\end{framesubsec}

\begin{framesubsec}[Anónimas]{Estructuras anónimas}
	\lstinputlisting[language=C]{csrc/struct_annom.c}
\end{framesubsec}

\begin{framesubsec}{Arrays y punteros}
	\lstinputlisting[language=C]{csrc/struct_ptr.c}
\end{framesubsec}

\begin{framesec}{Union}
	\begin{itemize}
		\item Una unión es un valor que tiene varias representaciones o
			formatos
		\item Estructura que permite guardar varios tipos de datos en la
			misma zona de memoria
	\end{itemize}
	\lstinputlisting[language=C, basicstyle=\scriptsize]{csrc/union.c}
\end{framesec}

\begin{framesubsec}{Ejemplo 1}
	\lstinputlisting[language=C]{csrc/union_e1.c}
\end{framesubsec}

\begin{framesubsec}{Ejemplo 2}
	\setlength\columnsep{1cm}
	\begin{multicols}{2}
		\lstinputlisting[language=C, basicstyle=\tiny]{csrc/union_e2.c}
	\end{multicols}
\end{framesubsec}

\begin{framesec}{Campos de bits}
	\begin{itemize}
		\item Característica de las estructuras y uniones que nos
			permite declarar campos de hasta un bit de longitud
		\item La memoria reservada es la que indica el tipo del campo
		\item Para acceder a nivel de bit se realizan numerosas
			operaciones por debajo
	\end{itemize}
	\vspace{2em}
	\begin{columns}[onlytextwidth]
	\column{0.5\textwidth}
		\lstinputlisting[language=C, frame=none, numbers=none]{csrc/bitfield.c}
	\column{0.4\textwidth}
		\ig{img/bitfields}{}{}
	\end{columns}
\end{framesec}

\begin{framesubsec}{Ejemplo}
	\lstinputlisting[language=C, basicstyle=\scriptsize]{csrc/bitfield_float.c}
\end{framesubsec}

\section{Enumerados}
\begin{framesubsec}[Macros]{Macros: El preprocesador de C}
	\begin{itemize}
		\item Preprocesador: Se ejecuta antes de compilar
		\item Lenguaje de \textbf{macros}
		\item Multiples usos:
		\begin{itemize}
			\item Declaración de constantes
			\item Pequeñas funciones y utilidades
			\item Compilación condicional de código
			\item Depuración
		\end{itemize}
	\end{itemize}
\end{framesubsec}

\begin{framesubsubsec}{Ejemplos}
	\lstinputlisting[language=C, basicstyle=\tiny]{csrc/macros.c}
\end{framesubsubsec}

\begin{framesubsec}{Enum}
	\lstinline[language=C]|enum estado_coche \{ARRANCADO, PARADO, EN_MARCHA, DETENIDO\};|

	\begin{itemize}
		\item \textbf{Tipo} formado por una lista de macros
		\item Las macros toman valores enteros de forma
			consecutiva
	\end{itemize}
\end{framesubsec}

\begin{framesubsubsec}{Ejemplo}
	\setlength\columnsep{1cm}
	\begin{multicols}{2}
		\lstinputlisting[language=C, basicstyle=\tiny]{csrc/enum.c}
	\end{multicols}
\end{framesubsubsec}

\ejframe{Estructuras}

\end{document}
