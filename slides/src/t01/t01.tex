\documentclass{mybeamer}

\institute{
	{\textsl{\large Tema 1}}
	\\[1em]
	\textbf{\Large Introducción e Historia de C}
}

\begin{document}

\begin{frame}
\titlepage
\end{frame}

\note[itemize]{
	\item Presentación de alumnos
	\item Recordar cambios en aulas y horarios:
		\url{http://1984.lsi.us.es/wiki-c}
}

\begin{frame}
\frametitle{Índice}
\tableofcontents
\end{frame}

\begin{framesec}{Inicios}
	\setbeamercovered{transparent}
	\centering
	\ig{img/ken_thompson}{}{0.2}
	\hspace{1em}
	\ig{img/dennis_ritchie}{}{0.2}
	\hspace{1em}
	\ig{img/brian_kernighan}{}{0.2}
	\\
	\begin{itemize}
		\item<2-> \textbf{Ken Thompson}
		\item<2-> \textbf{Dennis Ritchie}
		\item<2-> \textbf{Brian Kernighan}
		\note<2>{
			\begin{itemize}
				\item Ken Thompson: Izquierda
				\item Dennis Ritchie: Centro
				\item Brian Kernighan: Derecha
			\end{itemize}
			\begin{itemize}
				\item Ken Thompson: Diseña B junto con Dennis
					(simplificación de BCPL)
				\item Dennis Ritchie: Diseña C
				\item Brian Kernighan: Documenta C junto con
					Dennis
			\end{itemize}
		}
	\end{itemize}

	\begin{itemize}
		\item<3->  \textbf{Bell Labs} (de \textbf{AT\&T})
		\note<3>{
			\begin{itemize}
				\item Inicios ligados a \textit{Alexander
					Graham Bell}
				\item Al principio una división de
					\textbf{AT\&T}, ahora de Nokia
			\end{itemize}
		}

		\item<4-> Ensamblador y \textbf{B} insuficientes
			\textrightarrow\ diseñan \textbf{C}
		\note<4>{
			\textbf{B} fue creado por \textbf{Dennis Ritchie}
				con \textbf{Ken Thompson}
		}
		\item<5-> C fue desarrollado por \textbf{Dennis Ritchie} entre
			1969 y \textbf{1973}

		\item<6-> \textbf{Unix} reescrito en C (1973)
		\note<6>{Al principio estaba en su mayoría en ensamblador}
		\item<7-> En 1973 \textit{Brian Kernighan} y \textit{Dennis
			Ritchie} publican \textbf{The C Programming Language}
			(\textbf{K\&R}), que por muchos años sirvió como
			especificación informal del lenguaje.
		\item<8-> Posteriormente se añaden más funcionales a C y se
			estandariza.
	\end{itemize}
\end{framesec}

\begin{framesec}{Influencias}
	\ig{img/languages_h}{}{}
\end{framesec}

\begin{framesec}{¿Por qué C?}
	\centering
	\only<1| handout:1>{
	\begin{itemize}
		\item Simpleza
		\item Características de bajo nivel
		\item Madurez
		\item Eficiencia
		\item Portabilidad
		\item Numerosas bibliotecas y herramientas
	\end{itemize}
	}
	\only<2| handout:2>{
	\begin{itemize}
		\item Popularidad
	\end{itemize}
	\ig{img/tiobe}{1}{0.8}
	}
\end{framesec}

\note[itemize]{
	\item Manejo de memoria
	\item 44 años de madurez
	\item Compiladores muy buenos
}

\begin{framesec}{¿Para qué C?}
	\begin{itemize}
		\item \textbf{Ciencia:}
		\begin{itemize}
			\item Simulaciones
			\item Operaciones con grandes cantidades de tatos
		\end{itemize}
		\item \textbf{Sistemas Empotrados:}
		\begin{itemize}
			\item Sistemas Operativos en tiempo real
			\item Electrodomésticos, ascensores, automovilismo \ldots
		\end{itemize}
		\item \textbf{Robótica}
		\begin{itemize}
			\item Drones
			\item Robots humanoides
			\item Coches autónomos
		\end{itemize}
		\item \textbf{Medicina}
		\begin{itemize}
			\item Prótesis robóticas
			\item Equipamiento médico
		\end{itemize}
		\item \textbf{Sistemas Operativos}
	\end{itemize}
\end{framesec}

\begin{framesubsec}{Proyectos en C}
	\begin{itemize}
		\item Unix, GNU/Linux, kernel de MacOS y kernel de Windows
		\item Firefox y muchos otros exploradores (gumbo)
		\item Apache
		\item Gnome (GTK)
		\item Rover Curiosity (2.5 millones de lineas)
	\end{itemize}
\end{framesubsec}

\section{Sumer Of Code}
\begin{framesubsec}[GSOC]{Google Summer Of Code}
	\begin{minipage}[0.2\textheight]{\textwidth}
	\begin{columns}[c]
		\begin{column}{0.7\textwidth}
		\begin{itemize}
			\item Beca de Google para estudiantes
			\item Trabajas \textbf{3 meses} en un proyecto de
				\textbf{software libre}
			\item Experiencia
			\item Dinero: 5500\$
		\end{itemize}
		\vspace{2em}
		\url{https://summerofcode.withgoogle.com/}
		\end{column}

		\begin{column}{0.3\textwidth}
			\ig{img/gsoc}{}{}
		\end{column}
	\end{columns}
	\end{minipage}
\end{framesubsec}

\begin{framesubsec}{Outreachy}
	\ig{img/outreachy}{}{}
	\\[2em]
	\begin{itemize}
		\item Beca de Gnome para:
		\begin{itemize}
			\item mujeres
			\item grupos discriminados o con poca
				representación en el mundo tecnológico
			\item \textbf{Que no hayan participado antes ni
				en Outreachy ni en GSOC}
		\end{itemize}
		\item Trabajas \textbf{3 meses} en un proyecto de
			\textbf{software libre}
		\item Experiencia
		\item Dinero: 5500\$
	\end{itemize}
	\vspace{2em}
	Hasta el 22 Marzo
	\url{https://gnome.org/outreachy/}
\end{framesubsec}


\end{document}
