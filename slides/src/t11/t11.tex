\documentclass{mybeamer}
\usepackage{beramono} % For bold texttt

\institute{
	{\textsl{\large Tema 11}}
	\\[1em]
	\textbf{\Large Punteros a funciones}
}

\begin{document}
\begin{frame}
\titlepage
\end{frame}

\begin{frame}
\frametitle{Índice}
% \begin{multicols}{2}
	\tableofcontents
% \end{multicols}
\end{frame}

\begin{framesec}{Introducción}
	\centering
	\lstinline[language=C]|const char *(*ptr\_2\_func)(int, char *);|
	\vspace{2em}

	\begin{itemize}
		\item Guardan la dirección de una función
		\item Muy útiles en ciertas situaciones
		\item Sintaxis complicada (Consultar \url{http://cdecl.org/})
	\end{itemize}
\end{framesec}

\begin{framesec}[Ejemplo 1]{Ejemplo 1: Declaración, asignación y llamada}
	\lstinputlisting[language=C, basicstyle=\scriptsize]{csrc/ptrf.c}
\end{framesec}

\def\cmplst[#1]{
	\lstinputlisting[language=C, basicstyle=\scriptsize, linerange={#1}]{csrc/comparator.c}
}
\begin{framesec}[Ejemplo 2]{Ejemplo 2: Comparator}
	\only<1| handout:1>{                       % struct and main
		\cmplst[3-6]
		\cmplst[44-53]
	}
	\only<2| handout:2>{\cmplst[8-26]}         % comparators
	\only<3| handout:3>{\cmplst[33-34, 35-42]} % typedef and max
	\only<4| handout:4>{                       % main and out
		\cmplst[44-53]
		\lstinputlisting[language=bash, numbers=none, frame=none]{csrc/comparator.out}
	}
\end{framesec}

\def\arrylst[#1]{
	\lstinputlisting[language=C, basicstyle=\scriptsize, linerange={#1}]{csrc/ptrfarray.c}
}
\begin{framesec}[Ejemplo 3]{Ejemplo 3: Array}
	\only<1| handout:1>{\arrylst[1-8, 11-16, 18-27]}   % with switch
	\only<2| handout:2>{\arrylst[1-17, 25-27]}         % with ptr
\end{framesec}

\end{document}
