\documentclass{mybeamer}

\institute{
	{\textsl{\large Anexo}}
	\\[1em]
	\textbf{\Large Cuenta en GitHub}
}

\begin{document}
\begin{frame}
\titlepage
\end{frame}

\begin{frame}
\frametitle{Índice}
\tableofcontents
\end{frame}

\begin{framesec}{Cuenta en GitHub}
		\centering
		Elegimos un nombre de usuario, una contraseña e introducimos
		nuestro correo
		\vspace{1em}
		\hrule
		\vspace{1em}
		\ig{img/gh_ss/1}{}{0.6}
\end{framesec}

\begin{framesec}{Plan gratuito}
		\centering
		Nos aseguramos de que el plan gratuito está seleccionado y
		hacemos click en ``\textit{Finish sing up}''
		\vspace{1em}
		\hrule
		\vspace{1em}
		\ig{img/gh_ss/2}{}{0.7}
\end{framesec}

\begin{framesec}{Confirmar correo}
	\only<1>{
		\centering
		Debemos confirmar la dirección de correo
		\vspace{1em}
		\hrule
		\vspace{1em}
		\ig{img/gh_ss/3}{}{}
	}
	\only<2>{
		\centering
		Buscamos el correo de confirmación en nuestro buzón y hacemos
		click en ``\textsl{Verify email address}''
		\vspace{1em}
		\hrule
		\vspace{1em}
		\ig{img/gh_ss/4}{}{0.7}
	}
\end{framesec}

\begin{framesec}[Crear repo]{Crear repositorio}
		\centering
		¡Ya tenemos cuenta en GitHub! Ahora tenemos que crear un nuevo
		repositorio haciendo click donde apunta la flecha roja
		\vspace{1em}
		\hrule
		\vspace{1em}
		\ig{img/gh_ss/5}{}{}
\end{framesec}

\begin{framesubsec}{Nombre}
		\centering
		El nombre del repositorio tiene que tener el formato
		``\textbf{cursoc-\textless tu\_nombre\_compuesto\textgreater }''
		\vspace{1em}
		\hrule
		\vspace{1em}
		\ig{img/gh_ss/6}{}{0.7}
\end{framesubsec}

\begin{framesubsec}{Dirección}
		\centering
		Copiamos la dirección HTTPS de nuestro repositorio
		\vspace{1em}
		\hrule
		\vspace{1em}
		\ig{img/gh_ss/7}{}{0.9}
\end{framesubsec}

\end{document}
