\documentclass{mybeamer}

\institute{
	{\textsl{\large Proyecto}}
	\\[1em]
	\textbf{\Large Juego de la vida}
}

\begin{document}
\begin{frame}
\titlepage
\end{frame}

\begin{frame}
\frametitle{Índice}
\tableofcontents
\end{frame}

\begin{framesec}{Introducción}
	\centering
	\ig{img/logo}{}{0.3}
	\begin{itemize}
		\item juego de 0 jugadores
		\item Rejilla de células cuadradas como universo bidimensional
			ortogonal (infinito o no)
		\item Cada célula tiene dos estados (muerta o viva) e interactúa
			con sus 8 vecinas según unas reglas
		\item La regla más común es:
		\begin{itemize}
			\item \textbf{Nacimiento}: Una célula muerta con
				exactamente 3 vecinas vivas estará viva en la
				siguiente iteración
			\item \textbf{Supervivencia}: Una célula viva con 2 o 3
				vecinas vivas seguirá viva en la siguiente
				iteración, de lo contrario morirá.
		\end{itemize}
	\end{itemize}
\end{framesec}

\begin{framesec}{Ejemplo}
	\centering%
	\only<1| handout:1>{\ig{img/example1}{}{}}%
	\only<2| handout:2>{\ig{img/example2}{}{}}%
	\only<3| handout:3>{\ig{img/example3}{}{}}%
\end{framesec}

\begin{framesec}{Enlaces de interés}
	\begin{itemize}
		\item Más información:
			\\\href{https://es.wikipedia.org/wiki/Juego_de_la_vida}{es.wikipedia.org/wiki/Juego\_de\_la\_vida}
		\item Simulador (muy bueno):
			\\\href{http://golly.sourceforge.net/}{golly.sourceforge.net/}
		\item Simulador web:
			\\\href{{http://pmav.eu/stuff/javascript-game-of-life-v3.1.1/}}{pmav.eu/stuff/javascript-game-of-life-v3.1.1/}
	\end{itemize}
\end{framesec}

\end{document}
