\documentclass{mybeamer}
\usepackage{marvosym}

\institute{
	{\textsl{\large Tema 13}}
	\\[1em]
	\textbf{\Large Depuración}
}

\begin{document}
\begin{frame}
\titlepage
\end{frame}

\begin{frame}
\frametitle{Índice}
% \begin{multicols}{2}
	\tableofcontents
% \end{multicols}
\end{frame}

\begin{framesec}{Introducción}
	\begin{itemize}
		\item \textbf{¿Qué es depurar (debugging)?}: El proceso por el
			cual buscas y/o eliminas fallos (bugs) de tu programa

		\item \textbf{¿Qué es un depurador (debugger)?}: Una herramienta
			que facilita la depuración al permitir detener tu
			programa en cualquier instante y mirar que está pasando

		\item \textbf{¿Por qué es importante usarlo?}: . El método
			tradicional de los \texttt{printf} de depuración está
			bien, pero se queda corto en cuanto el programa crece en
			complejidad. En este punto, el depurador se vuelve la
			manera más eficiente y fácil de depurarlo.
	\end{itemize}
\end{framesec}

\begin{framesec}{GDB}
	\begin{itemize}
		\item Depurador del proyecto GNU
		\item Un proyecto muy maduro (30 años)
		\item Repleto de utilidades para depurar
		\item Capaz de depurar varios lenguajes (\textbf{C}, C++, Ada,
			Objective-C, Free Pascal, Fortran, Java \ldots)
		\item Soporta muchas arquitecturas (x86, x86-64, ARM, AVR,
			\ldots)
		\item Sistema cliente-servidor que permite depuración remota
		\item Base de muchos depuradores gráficos actuales (CodeBlocs,
			Eclipse, NetBeans, VisualStudio, QtCreator, \ldots)
	\end{itemize}
\end{framesec}

\begin{framesec}{Cómo utilizar GDB}
	\begin{enumerate}
		\item Compilar con símbolos de depuración \texttt{gcc -g main.c
			-o miprog} (\texttt{-ggdb} para símbolos específicos de
			gdb)
		\item Arrancar nuestro programa con gdb: \texttt{gdb miprog}
		\item Utilizar los comandos de gdb para depurar
	\end{enumerate}
\end{framesec}

\begin{framesec}{Comandos}
	\centering \Huge \bfseries COMANDOS
\end{framesec}

\begin{framesubsec}{Inicio}
	El programa se inicioa con el comando \texttt{run}. Si nuestro programa
	recibe parámetros, se los pasamos a \texttt{run}:

	\begin{center}
		\texttt{\Large (gdb) run -w 10 -h 15}
	\end{center}
\end{framesubsec}

\begin{framesubsec}{Breakpoints}
	\only<1| handout:1>{
	Un \textbf{breakpoint} define el momento en el que se va a detener
	nuestro programa para permitirnos examinarlo.
	\\[1em]
	Podemos elegir que se detenga cuando:
	\begin{itemize}
		\item Alcance cierta línea de código:\\
			\texttt{(gdb) break main.c:15}
		\item Entre en una función determinada:\\
			\texttt{(gdb) break world\_alloc}
		\item Se modifique cierta variable:\\
			\texttt{(gdb) watch contador}
	\end{itemize}
	}

	\only<2| handout:2>{
	Otros comandos:
	\begin{itemize}
		\item Listar los breakpoints (y watchpoints):\\
			\texttt{(gdb) info breakpoints}
		\item Elimiar breakpoint:\\
			\texttt{(gdb) delete <num que sale al listar>}
		\item Elimiar todos los breakpoint:\\
			\texttt{(gdb) delete breakpoints}
	\end{itemize}
	}
\end{framesubsec}

\begin{framesubsec}{Paso a paso}
	Tenemos varios comandos para realizar una ejecución ``paso a paso'' de
	nuestro programa:

	\begin{itemize}
		\item Ejecutar hasta el siguiente breakpoint:\\
			\texttt{(gdb) continue}
		\item Ejecutar una línea (sin entrar en funciones):\\
			\texttt{(gdb) next}
		\item Ejecutar una línea (entrado en funciones):\\
			\texttt{(gdb) step}
		\item Salir de la función actual:\\
			\texttt{(gdb) finish}
	\end{itemize}
\end{framesubsec}

\begin{framesubsec}{Estado del programa}
	\begin{itemize}
		\item Ver una variable:\\
			\texttt{(gdb) print contador}
		\item Ver una estructura:\\
			\texttt{(gdb) print *world}
		\item Ver el resultado de una expresión:\\
			\texttt{(gdb) print size\_x * size\_y}
		\item Ver el resultado de una función:\\
			\texttt{(gdb) print world\_get\_cell(w, 1, 2)}
		\item Ver las variables locales:\\
			\texttt{(gdb) info locals}
		\item Ver algunas línas de código de alrededor:\\
			\texttt{(gdb) list}
		\item Ver la pila de llamadas:\\
			\texttt{(gdb) backtrace}
	\end{itemize}
\end{framesubsec}

\begin{framesubsec}{Otros comandos útiles}
	\begin{itemize}
		\item Ver clases de comandos:\\
			\texttt{(gdb) help}
		\item Ver comandos de un clase (por ejemplo, breakpoints):\\
			\texttt{(gdb) help breakpoints}
		\item Ver ayuda sobre un comando (por ejemplo, break):\\
			\texttt{(gdb) help break}
		\item Para salir:\\
			\texttt{(gdb) quit}
		\item Puedes abreviar un comando siempre que no sea ambiguo:\\
			\texttt{(gdb) help n} == \texttt{(gdb) help next}
		\item Para ejecutar el comando anterior pulsar ENTER
		\item Para autocompletar (comandos, nombres de funciones y
			variables, etc) pulsar TAB
		\item Chuleta: \url{http://lab46.corning-cc.edu/_media/opus/fall2014/mgardne8/70120637.png}
	\end{itemize}
\end{framesubsec}

\end{document}
