\documentclass{mybeamer}
\usepackage{marvosym}

\institute{
	{\textsl{\large Tema 16}}
	\\[1em]
	\textbf{\Large Fork}
}

\begin{document}
\begin{frame}
\titlepage
\end{frame}

\begin{frame}
\frametitle{Índice}
% \begin{multicols}{2}
	\tableofcontents
% \end{multicols}
\end{frame}

\begin{framesec}{Paralelismo}
	\begin{itemize}
		\item A veces es \textit{necesario} que nuestro programa realice
			varias tareas \textbf{a la vez}
		\item Otras veces es \textit{interesante} dividir el trabajo en
			partes independientes que pueden realizarse
			simultáneamente (\textbf{en paralelo}). Esto puede
			mejorar enormemente la eficiencia de nuestro programa,
			incluso si nuestro hardware no tiene la capacidad de
			ejecutar varios procesos a la vez
		\item \textbf{No siempre es lo mejor}. Paralelizar un programa
			tiene un coste de rendimiento que, a veces, puede
			superar a la mejora obtenida.
	\end{itemize}
\end{framesec}

\begin{framesec}{Procesos}
	\begin{itemize}
		\item En Unix, hilos y procesos son esencialmente lo mismo
		\item Todo proceso tiene asociado un número que lo identifica:
			\textbf{PID}(\textit{Process IDentification})
		\item En Unix todos procesos se crean mediante la llamada al
			sistema \lstinline[language=C]{pid\_t fork(void);}
	\end{itemize}
\end{framesec}

\begin{framesec}{\texttt{fork()}}
	\begin{itemize}
		\item En el momento en que un programa llama a \texttt{fork()},
			se duplica.
		\item A partir de entonces existirán dos procesos (padre e hijo)
			con el mismo código, las mismas variables, etc.
		\item El proceso hijo tendrá una copia de la memoria del padre
			en el instante en el que se llamó a \texttt{fork()}.
		\item La memoria es independiente. Si uno de los procesos
			modifica una variable, no afecta a la variable del otro
			proceso. (Existe la opción de que ambos procesos
			compartan la memoria, pero se excede del propósito de
			este curso)
	\end{itemize}
\end{framesec}

\begin{framesubsec}{Ejemplo}
	\only<1| handout:1>{\lstinputlisting[language=C, basicstyle=\tiny]{csrc/fork.c}}
	\vspace{1em}
	\only<2| handout:2>{\lstinputlisting[language=bash, frame=none, numbers=none]{csrc/fork.out}}
\end{framesubsec}

\begin{framesubsec}{Zombies}
	\begin{itemize}
		\item Cuando un proceso hijo muere, no muere del todo. Permanece
			en un estado llamado \textbf{zombie} que permite al
			padre recuperar información de él aun cuando ha
			terminado (el código de error por ejemplo)
		\item El sistema puede llegar a colapsarse si se crean mucho
			procesos zombies. Conclusión: Siempre llamar a
			\texttt{wait()})
		\item Si un proceso padre muere antes que su hijo (no ha llamado
			a (\texttt{wait()}), el comportamiento depende del SO:
		\begin{itemize}
			\item El padre del proceso hijo pasa a ser el proceso
				\textbf{init} (PID = 1)
			\item El hijo es matado también inmediatamente
		\end{itemize}
	\end{itemize}
\end{framesubsec}

\begin{framesec}{Señales}
	\begin{itemize}
		\item Son un medio para que los procesos puedan interaccionar
			entre sí. Un proceso envía una señal y el otro la
			recibe.
		\item Existen muchos tipos de señales distintas (\texttt{man 7
			signal})
		\item Para reaccionar ante la recepción de una señal debemos
			instalar un \textbf{callback}
	\end{itemize}
\end{framesec}

\begin{framesubsec}{Ejemplo}
	\lstinputlisting[language=C, basicstyle=\tiny]{csrc/signal.c}
\end{framesubsec}

\end{document}
