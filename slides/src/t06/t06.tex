\documentclass{mybeamer}
\usepackage{tikz}
\newcommand{\tikzmark}[2]{%
	     \tikz[overlay,remember picture] \node[text=black, inner sep=2pt] (#1) {#2};
}

\institute{
	{\textsl{\large Tema 6}}
	\\[1em]
	\textbf{\Large Reserva dinámica de memoria}
}

\begin{document}
\begin{frame}
\titlepage
\end{frame}

\begin{frame}
\frametitle{Índice}
% \begin{multicols}{2}
	\tableofcontents
% \end{multicols}
\end{frame}

\begin{framesec}{Mapa de memoria}
	\note{
		\begin{itemize}
			\item Explicar \textbf{malloc} y \textbf{free}
			\item Explicar que después de cada malloc hay que
				\textbf{comprobar siempre}
		\end{itemize}
	}
	\centering
	\begin{columns}[onlytextwidth]
		\begin{column}{0.6\textwidth}
			\begin{itemize}
				\item \textbf{Heap}: Se almacena la memoria
					reservada \textbf{dinámicamente} con
					\textbf{malloc}
				\item \textbf{Stack}: Se almacenan las
					\textbf{variables locales} de cada
					llamada a función
				\item \textbf{Globales}: Todas las variables
					globales
				\item \textbf{Constantes}: Todas las constantes
					(números, cadenas, etc)
				\item \textbf{Código}: El programa en sí
			\end{itemize}
		\end{column}

		\begin{column}{0.3\textwidth}
			\ig{img/mem_map}{}{}
		\end{column}
	\end{columns}
\end{framesec}

\begin{framesubsec}{Ejemplo}
	\note{
		\begin{itemize}
			\item \lstinline[language=C]|\#include <stdlib.h>| para
				el \textbf{malloc} y el \textbf{free}
			\item Explicar \textbf{static}
		\end{itemize}
	}
	\centering
	\begin{columns}[onlytextwidth]
		\begin{column}{0.6\textwidth}
			\lstinputlisting[language=C, basicstyle=\tiny]{csrc/mm_example.c}
		\end{column}

		\begin{column}{0.38\textwidth}
			\only<1| handout:1>{\ig{img/mm_e1}{}{}}
			\only<2| handout:2>{\ig{img/mm_e2}{}{}}
			\only<3| handout:3>{\ig{img/mm_e3}{}{}}
			\only<4| handout:4>{\ig{img/mm_e4}{}{}}
			\only<5| handout:5>{\ig{img/mm_e5}{}{}}
		\end{column}
	\end{columns}
\end{framesubsec}

\begin{framesec}{Stack Overflow}
	El uso de funciones recursivas puede ocasionar un agotamiento de la
	pila

	\begin{columns}[onlytextwidth]
		\begin{column}{0.6\textwidth}
			\lstinputlisting[language=C, basicstyle=\tiny]{csrc/stack_overflow.c}
		\end{column}

		\begin{column}{0.38\textwidth}
			\only<1| handout:1>{\ig{img/so_1}{}{}}
			\only<2| handout:2>{\ig{img/so_2}{}{}}
			\only<3| handout:3>{\ig{img/so_3}{}{}}
			\only<4| handout:4>{\ig{img/so_4}{}{}}
		\end{column}
	\end{columns}
\end{framesec}

\begin{framesec}{Fragmentación}
	Fragmentos pequeños de memoria libre entre bloques de memoria reservada.

	\begin{columns}[onlytextwidth]
		\begin{column}{0.6\textwidth}
			\lstinputlisting[language=C, basicstyle=\tiny]{csrc/fragm.c}
		\end{column}

		\begin{column}{0.3\textwidth}
			\only<1| handout:1>{\ig{img/frag_1}{}{}}
			\only<2| handout:2>{\ig{img/frag_2}{}{}}
			\only<3| handout:3>{\ig{img/frag_3}{}{}}
		\end{column}
	\end{columns}
\end{framesec}

\section[arrays multiD]{Reserva de arrays multidimensionales}
\begin{framesubsec}[Forma 1]{Forma 1 (la mala)}
	\only<1| handout:1>{
		Un malloc por cada fila del array (array de punteros)

		\lstinputlisting[language=C, basicstyle=\tiny]{csrc/array_filas.c}
	}

	\only<2| handout:2>{
		\centering
		No se garantiza continuidad entre los bloques reservados
		\vspace{1em}

		\ig{img/array_filas}{}{}
	}
\end{framesubsec}

\begin{framesubsec}[Forma 2]{Forma 2 (la buena)}
	\lstinputlisting[language=C, basicstyle=\tiny]{csrc/array_bloque.c}
\end{framesubsec}

\begin{framesec}{Uso tras liberación}
	\lstinputlisting[language=C, basicstyle=\tiny]{csrc/use_empty.c}
\end{framesec}

\begin{framesec}[valgrind]{Depuración con valgrind}
	\begin{itemize}
		\item Valgrind es una herramienta que nos avisa de los errores
			cometidos en el manejo de memoria
		\item Un \textbf{leak} o fuga de memoria es un error de
			programación que hace que la memoria reservada no se
			libere cuándo ya no se usa. Un programa que no libere
			correctamente la memoria reservada puede acabar
			consumiendo toda la memoria del sistema y dejarlo
			inutilizado.
		\item Para que valgrind nos muestre más información podemos
			hacer dos cosas:
		\begin{enumerate}
			\item Compilar con símbolos de depuración (\texttt{gcc
				-g})
			\item Ejecutar valgrind con el flag
				\texttt{--leak-check=full}
		\end{enumerate}
	\end{itemize}
\end{framesec}

\begin{framesubsec}{Ejemplo}
	\only<1| handout:1>{
		\lstinputlisting[language=C, basicstyle=\tiny]{csrc/valgrind.c}
	}

	\only<2| handout:2>{
		\centering
		\lstinline[language=bash]|gcc -g valgrind.c -o valg_ej|
		\lstinline[language=bash]|valgrind --leak-check=full valg_ej|

		\lstinputlisting[basicstyle=\tiny]{csrc/valgrind.bash}
	}
\end{framesubsec}

\ejframe{Reserva dinámica de memoria}

\end{document}
