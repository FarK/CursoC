\documentclass{mybeamer}

\institute{
	{\textsl{\large  Anexo}}
	\\[1em]
	\textbf{\Large Flujo de trabajo}
	\\
	{\normalsize pull-request en GitHub}
}

\begin{document}

\begin{frame}
\titlepage
\end{frame}

\begin{frame}
\frametitle{Índice}
\begin{multicols}{2}
	\tableofcontents
\end{multicols}
\end{frame}

\begin{framesec}[Cambios]{Crea y revisa tus cambios}
	\ig{img/ss_mod/01.png}{}{}
\end{framesec}

\begin{framesec}[pull request]{Crea un nuevo pull request}
	En la pestaña \textit{Pull Request} pulsa \textit{New pull request}
	\ig{img/ss_mod/02.png}{}{}
\end{framesec}

\begin{framesec}[Elige la rama]{Elige la rama sobre la que aplicar tus cambios}
	\begin{enumerate}
		\item click en \textit{compare across forks}
		\item elegir la rama del profesor como \textit{base fork}
		\item elegir tu rama como \textit{head fork}
		\item click en \textit{Create pull request}
	\end{enumerate}
	\centering
	\ig{img/ss_mod/03.png}{}{0.6}
\end{framesec}

\begin{framesec}[Título y comentario]{Añade un título y un comentario}
	\ig{img/ss_mod/04.png}{}{}
\end{framesec}

\begin{framesec}{Solicitado con éxito}
	\ig{img/ss_mod/05.png}{}{}
\end{framesec}

\begin{framesec}[Revisiones]{Espera las revisiones del profesor}
	\ig{img/ss_mod/06.png}{}{}
\end{framesec}

\begin{framesec}{Arreglos}
	\only<1-2>{
		Crea un nuevo commit (o varios) para solucionar las correcciones
		que te hayan pedido
	}
	\only<1>{\ig{img/ss_mod/07.png}{}{}}
	\only<2>{\ig{img/ss_mod/08.png}{}{}}
	\only<3>{
		El nuevo commit deberá salir en la página de tu pull request
		\ig{img/ss_mod/09.png}{}{}
	}
\end{framesec}

\begin{framesec}[Aceptado]{Espera a nuevas revisiones o a que sea aceptado}
	\ig{img/ss_mod/10.png}{}{}
\end{framesec}

\begin{framesec}{Network}
	Puedes ver tu pull request gráficamente en la pestaña \textit{network}
	\ig{img/ss_mod/11.png}{}{}
\end{framesec}

\end{document}
