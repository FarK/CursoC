\documentclass{mybeamer}

\institute{
	{\textsl{\large Proyecto}}
	\\[1em]
	\textbf{\Large Juego de la vida}
}

\begin{document}
\begin{frame}
\titlepage
\end{frame}

\begin{frame}
\frametitle{Índice}
\tableofcontents
\end{frame}

\begin{framesec}{Introducción}
	\centering
	\ig{img/gol}{}{0.3}
	\begin{itemize}
		\item juego de 0 jugadores
		\item Rejilla de células cuadradas como universo bidimensional
			ortogonal (infinito o no)
		\item Cada célula tiene dos estados (muerta o viva) e interactúa
			con sus 8 vecinas según unas reglas
		\item La regla más común es:
		\begin{itemize}
			\item \textbf{Nacimiento}: Una célula muerta con
				exactamente 3 vecinas vivas estará viva en la
				siguiente iteración
			\item \textbf{Supervivencia}: Una célula viva con 2 o 3
				vecinas vivas seguirá viva en la siguiente
				iteración, de lo contrario morirá.
		\end{itemize}
	\end{itemize}
\end{framesec}

\end{document}
