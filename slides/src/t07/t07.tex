\documentclass{mybeamer}
\usepackage{tikz}
\newcommand{\tikzmark}[2]{%
	     \tikz[overlay,remember picture] \node[text=black, inner sep=2pt] (#1) {#2};
}

\institute{
	{\textsl{\large Tema 7}}
	\\[1em]
	\textbf{\Large Objetos}
}

\begin{document}
\begin{frame}
\titlepage
\end{frame}

\begin{frame}
\frametitle{Índice}
% \begin{multicols}{2}
	\tableofcontents
% \end{multicols}
\end{frame}

\begin{framesec}[Qué es?]{¿Qué es un objeto?}
	\begin{itemize}
		\item Un objeto es una \textit{entidad} con:
		\begin{itemize}
			\item Un \textbf{estado}: datos que guarda
			\item Un \textbf{comportamiento}: métodos que
				interactúan con ese estado.
		\end{itemize}
		\item Un objeto es una \textbf{instancia} de una \textbf{clase}
			específica
		\item Un clase puede \textbf{heredar} propiedades de otras
	\end{itemize}
\end{framesec}

\begin{framesec}[Ejemplo]{Ejemplo de objeto en C}
	\centering
	{\Huge Ejemplo de objeto}

	{\Large persona}
\end{framesec}

\begin{framesubsec}{Ocultación}
	\note{
		\begin{itemize}
			\item Mediante \textbf{\textit{forward declaration}}
				(declaración adelantada) conseguimos que los
				atributos de la clase sean \textbf{privados}
			\item En el main no podemos:
			\begin{itemize}
				\item Conocer ni acceder a los miembros de la
					misma
				\item Reservar memoria para la estructura, ni
					estática ni dinámicamente. Esto es
					porque no conocemos el tamaño que ocupa
			\end{itemize}
		\end{itemize}
	}

	\begin{columns}[onlytextwidth]
		\begin{column}{0.4\textwidth}
			\centering
			person.h
			\lstinputlisting[language=C, basicstyle=\tiny, linerange={1-4,20-21}]{csrc/person.h}
		\end{column}

		\begin{column}{0.4\textwidth}
			\centering
			person.c
			\lstinputlisting[language=C, basicstyle=\tiny, linerange={1-2,7-8,11-24}]{csrc/person.c}
		\end{column}
	\end{columns}

	\centering
	main.c
	\lstinputlisting[language=C, basicstyle=\tiny, linerange={3-7,22-24}, xleftmargin=0.3\textwidth, linewidth=0.7\textwidth]{csrc/main.c}
\end{framesubsec}

\begin{framesubsec}[flags]{Flags}
	person.c
	\lstinputlisting[language=C, basicstyle=\scriptsize, linerange={9-10}, linewidth=1.02\textwidth]{csrc/person.c}
	\vspace{2em}
	Necesitamos saber si un atributo ha sido o no establecido para:
	\begin{itemize}
		\item Saber si el valor que guarda o no es válido
		\item Saber si se ha reservado memoria y hay que liberarla
		\item Imprimir o no los atributos
		\item \ldots
	\end{itemize}
\end{framesubsec}

\begin{framesubsec}{Reserva}
	\note{Hay que añadir el malloc y el free al *.h también}
	person.c
	\lstinputlisting[language=C, basicstyle=\tiny, linerange={26-41}]{csrc/person.c}
	main.c
	\lstinputlisting[language=C, basicstyle=\tiny, linerange={5-9,21,24}]{csrc/main.c}
\end{framesubsec}

\begin{framesubsec}{Setters}
	\lstinputlisting[language=C, basicstyle=\tiny, linerange={43-72}]{csrc/person.c}
\end{framesubsec}

\begin{framesubsec}{Getters}
	\lstinputlisting[language=C, basicstyle=\scriptsize, linerange={74-96}]{csrc/person.c}
\end{framesubsec}

\begin{framesubsec}{Print}
	\lstinputlisting[language=C, linerange={98-112}]{csrc/person.c}
\end{framesubsec}

\end{document}
